\documentclass[12pt,a4paper]{article}
\usepackage[utf8]{inputenc}
\usepackage[T1]{fontenc}
\usepackage[brazil]{babel}
\usepackage{geometry}
\usepackage{setspace}
\usepackage{hyperref}
\usepackage{mathptmx} 
\usepackage{microtype}

\geometry{a4paper, top=2.5cm, bottom=2.5cm, left=2cm, right=2cm}
\setstretch{1.5} 

\begin{document}

\begin{center}
    {\fontsize{14pt}{16pt}\selectfont \textbf{INTERFACES LÚDICO-TECNOLÓGICAS: 
    A CONVERGÊNCIA ENTRE GAMIFICAÇÃO E RECURSOS DIGITAIS NO PROCESSO EDUCATIVO}}
\end{center}

\vspace{2em}


\noindent
\hfill
\begin{minipage}[t]{0.90\textwidth}
\raggedleft
Camilly Vitória Barbosa dos Santos \\
Análise e Desenvolvimento de Sistemas – Faculdade de Educação Tecnológica do Estado do Rio de Janeiro, FAETERJ-Rio \\
camilly.23204708360034@faeterj-rio.edu.br \\[1em]

Silvia Soares de Oliveira \\
Análise e Desenvolvimento de Sistemas – Faculdade de Educação Tecnológica do Estado do Rio de Janeiro, FAETERJ-Rio \\
silvia.23104708360030@faeterj-rio.edu.br
\end{minipage}

\vspace{2em}

\noindent
\begin{minipage}[t]{0.48\textwidth}
\raggedright
\textbf{AT06: Tecnologias Digitais na Educação}
\end{minipage}

\vspace{1em}

\noindent
\textbf{Introdução:} A presença crescente das tecnologias digitais no cotidiano dos estudantes tem exigido que a educação acompanhe essa evolução, tornando-se mais dinâmica, personalizada e conectada com a realidade dos alunos. Nesse cenário, a gamificação surge como estratégia ao aplicar elementos e mecânicas de jogos em ambientes de aprendizagem. Essa abordagem contribui para aumentar o engajamento, a motivação e o interesse pelo conteúdo escolar. Ao unir o lúdico ao digital, essa abordagem transforma a sala de aula em um espaço interativo e participativo, no qual os estudantes assumem um papel mais ativo em sua aprendizagem. \textbf{Objetivo:} Este trabalho tem como objetivo analisar de que forma a aplicação de elementos lúdicos, inspirados em mecânicas de jogos, integrados às tecnologias digitais, pode contribuir para tornar o processo de ensino e aprendizado mais eficaz e alinhado às demandas educacionais contemporâneas. Busca-se compreender os impactos dessa abordagem, destacando a importância da mediação pedagógica e da adaptação das estratégias ao perfil dos estudantes. \textbf{Metodologia:} Foi utilizada uma abordagem qualitativa, baseada em revisão bibliográfica. Foram analisadas diversas fontes, com ênfase em estudos que abordam a aplicação de elementos lúdicos e tecnologias digitais no contexto educacional. Priorizaram-se materiais que apresentassem dados, experiências práticas e análises sobre os efeitos dessas estratégias na motivação dos estudantes, no engajamento com o conteúdo e no desenvolvimento de habilidades cognitivas e socioemocionais. \textbf{Resultados:} Os resultados obtidos indicam um consenso positivo quanto ao potencial da incorporação de elementos de jogos no contexto educacional. A presença de elementos de jogos e recursos digitais contribui significativamente para o aumento da motivação e da participação dos alunos, ao transformar tarefas comuns em experiências mais atrativas e imersivas. Além disso, destaca-se a importância do cuidado na aplicação dessas estratégias, ressaltando a necessidade de adequação ao público-alvo e da atenção pedagógica por parte dos professores. Essa mediação é essencial para que o aprendizado seja adaptado ao estilo de cada estudante, bem como para garantir a oferta de feedback contínuo durante o progresso. Isso favorece a manutenção do foco, da autoestima e encoraja os alunos a sair da zona de conforto, buscar soluções criativas e colaborar com os colegas. \textbf{Conclusão:} A incorporação de abordagens lúdicas mediadas por tecnologias digitais representa uma alternativa para repensar as práticas pedagógicas. Quando utilizadas de forma intencional e bem planejada, essas estratégias não apenas facilitam a assimilação de conteúdos, mas também promovem o desenvolvimento de competências como autonomia, criatividade, resolução de problemas e cooperação entre os estudantes. Assim, o uso consciente da gamificação pode transformar a aprendizagem em uma experiência mais inclusiva e alinhada às demandas da educação contemporânea.

\vspace{1em}

\noindent\textbf{Palavras-chave:} Gamificação; Mediação pedagógica; Métodos de ensino; Tecnologias digitais.

\end{document}
