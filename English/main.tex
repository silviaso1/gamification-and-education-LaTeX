\documentclass[12pt,a4paper]{article}
\usepackage[utf8]{inputenc}
\usepackage[T1]{fontenc}
\usepackage[english]{babel}
\usepackage{geometry}
\usepackage{setspace}
\usepackage{hyperref}
\usepackage{mathptmx} 
\usepackage{microtype}

\geometry{a4paper, top=2.5cm, bottom=2.5cm, left=2cm, right=2cm}
\setstretch{1.5} 

\begin{document}

\begin{center}
    {\fontsize{14pt}{16pt}\selectfont \textbf{LUDIC-TECHNOLOGICAL INTERFACES: 
    THE CONVERGENCE BETWEEN GAMIFICATION AND DIGITAL RESOURCES IN THE EDUCATIONAL PROCESS}}
\end{center}

\vspace{2em}


\noindent
\hfill
\begin{minipage}[t]{0.90\textwidth}
\raggedleft
Camilly Vitória Barbosa dos Santos \\
Análise e Desenvolvimento de Sistemas – Faculdade de Educação Tecnológica do Estado do Rio de Janeiro, FAETERJ-Rio \\
camilly.23204708360034@faeterj-rio.edu.br \\[1em]

Silvia Soares de Oliveira \\
Análise e Desenvolvimento de Sistemas – Faculdade de Educação Tecnológica do Estado do Rio de Janeiro, FAETERJ-Rio \\
silvia.23104708360030@faeterj-rio.edu.br
\end{minipage}

\vspace{2em}

\noindent
\begin{minipage}[t]{0.48\textwidth}
\raggedright
\textbf{AT06: Digital Technologies in Education}
\end{minipage}

\vspace{1em}

\noindent
\textbf{Introduction:} The growing presence of digital technologies in students' daily lives has demanded that education keeps pace with this evolution, becoming more dynamic, personalized, and connected to students' realities. In this scenario, gamification emerges as a strategy by applying game elements and mechanics in learning environments. This approach contributes to increasing engagement, motivation, and interest in the school content. By combining playfulness with digital resources, this approach transforms the classroom into an interactive and participatory space, where students take a more active role in their learning. \textbf{Objective:} This study aims to analyze how the application of playful elements, inspired by game mechanics, integrated with digital technologies, can contribute to making the teaching and learning process more effective and aligned with contemporary educational demands. It seeks to understand the impacts of this approach, highlighting the importance of pedagogical mediation and the adaptation of strategies to students' profiles. \textbf{Methodology:} A qualitative approach based on a literature review was used. Several sources were analyzed, with an emphasis on studies addressing the application of playful elements and digital technologies in the educational context. Materials presenting data, practical experiences, and analyses of the effects of these strategies on student motivation, engagement with content, and the development of cognitive and socio-emotional skills were prioritized. \textbf{Results:} The results indicate a positive consensus regarding the potential of incorporating game elements in the educational context. The presence of game elements and digital resources significantly contributes to increased student motivation and participation by transforming ordinary tasks into more attractive and immersive experiences. Furthermore, the importance of careful application of these strategies is highlighted, emphasizing the need to adapt to the target audience and the teacher's pedagogical attention. Such mediation is essential for tailoring learning to each student's style and ensuring continuous feedback throughout their progress. This favors focus maintenance, self-esteem, and encourages students to step out of their comfort zone, seek creative solutions, and collaborate with peers. \textbf{Conclusion:} The incorporation of playful approaches mediated by digital technologies represents an alternative for rethinking pedagogical practices. When used intentionally and well-planned, these strategies not only facilitate content assimilation but also promote the development of competencies such as autonomy, creativity, problem-solving, and student collaboration. Thus, the conscious use of gamification can transform learning into a more inclusive experience aligned with contemporary educational demands.

\vspace{1em}

\noindent\textbf{Keywords:} Gamification; Pedagogical mediation; Teaching methods; Digital technologies.

\end{document}
